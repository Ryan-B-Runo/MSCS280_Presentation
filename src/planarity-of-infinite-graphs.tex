The graph $K_{1, n-1}$, the complete star of order n, is a graph defined as planar for all values of n.

\subsection{Simple Construction of: $K_{1, n-1}$} $\\$
For a graph $K_{1, n-1},\ n \geq 2$, it follows that in radial coordinates, all vertices must exist at distinct degrees.
That is, $\left\{ \exists\ \theta_i\ \forall\ 0\leq i, k < n-1 \in \Z\ \lvert\ \theta_i \neq \theta_k \text{ and } \forall\ \theta_i, \exists\text{ radius }r_i \right\}$.
Further, may it be assumed, for any value $n$, the distribution such that all values of $\theta_i$ are maximally spread from adjacent degrees: $\theta_{\text{mod}(i\pm 1, n-1)}$, is $\theta_i = \frac{2\pi}{n-1} i$.
It follows, the non-origin vertices must then be:
\[v_i = \left(r_i \cos\left(\frac{2\pi}{n-1} i\right), r_i \sin\left(\frac{2\pi}{n-1}i\right)\right);\ r_i \neq 0\]
with edges:
\[E\left(v_{\left(0,0\right)}, v_i\right) = \left(t\cdot r_i \cos\left(\frac{2\pi}{n-1} i \right), t\cdot r_i\sin\left(\frac{2\pi}{n-1} i\right) \right);\ 0\leq t\leq1\in \R\]
Thus, by construction, $K_{1, n-1}$ $n$-finite is planar.
\begin{figure}[htbp]
    \centering %\linewidth or scale=.3
    \includegraphics[height=.25\textheight]{nickScreenshot}
    \caption{$K_{1, n-1}$ of monotonically increasing radii}
    \label{fig:screenshot}
\end{figure}

\subsection{Planarity of $K_{1, \infty}$}
$\\$
The definition of planar requires edges to intersect only at their endpoints.\\
May it be self-evident that for any sequence of radii $r_i '$, there exists another orientation of radii $r_i$ that is monotonically increasing.
That is, $r_0\leq r_1 \leq r_2 \leq \cdots \leq r_n$.
Pick any two distinct vertices in $K_{1, n-1}$.\\
That is, let $\{k, c \in \Z\ \lvert\ 0\leq k<n-1,\ 0<c<n-1-k\}$.
It follows that the edges created from the images of the vertices $v_k$ and $v_{k+c}$ may only intersect when the radii are the same.
That is, $t_{k+c} = \frac{r_{k}}{r_{k+c}} t_{k}\ \lvert\ 0 < \frac{r_k}{r_{k+c}} t_{k} \leq 1 $.\\
May we evaluate $E\left(v_{\left(0, 0\right)}, v_i\right) = E\left(v_{\left(0, 0\right)}, v_{i+c}\right)$:
\[t_k\cdot r_k e^{i\frac{2\pi}{n-1}k} = t_{k+c}\cdot r_{k+c}\ e^{i\frac{2\pi}{n-1}(k+c)}\]
It follows that:
\begin{align*}
    t_k\cdot r_k e^{i\frac{2\pi}{n-1}k} - t_{k+c}\cdot r_{k+c}\ e^{i\frac{2\pi}{n-1}\left(k+c\right)} &= 0 \\
    r_{k+c} \cdot \left(\frac{r_k}{r_{k+c}}t_k e^{i\frac{2\pi}{n-1} k} - t_{k+c} e^{i\frac{2\pi}{n-1}} \left(k+c\right)\right) &= 0 \\
    r_{k+c} \cdot \left(t_{k+c} e^{i\frac{2\pi}{n-1} k} - t_{k+c} e^{i\frac{2\pi}{n-1} \left(k+c\right)}\right) &= 0 \\
    t_{k+c} \cdot r_{k+c} \cdot \left(e^{i\frac{2\pi}{n-1} k} - e^{i\frac{2\pi}{n-1} k} e^{i\frac{2\pi}{n-1}c}\right) &= 0 \\
    t_{k+c} \cdot r_{k+c} \cdot e^{i\frac{2\pi}{n-1} k} \cdot \left(1-e^{i\frac{2\pi}{n-1} c}\right) &= 0
\end{align*}
By the definition of Planar, an intersection at $t_{k+c}=0$ maintains planarity,
$r_{k+c}$ is defined to be strictly greater than 0, and $e^{i\frac{2\pi}{n-1} k}$ is a point on the unit circle.\\\\
Consider however:
\[1 - e^{i\frac{2\pi}{n-1}c} = 0\]
Then:
\begin{align*}
    e^{i\frac{2\pi}{n-1} c} &= 1 \\
    \frac{2\pi}{n-1} c &= 0 + 2\pi m, m\in \Z \\
    c &= m\cdot(n-1)
\end{align*}
Under the domain of $c$, $0<c<n-1-k$, $m$ can only ever be $0$ or $1$.
Notice further, $c$ has two strict inequalities for both $0$ and the largest difference of index, $n-1$.
That is, no finite $n$ allows for the fulfillment of the above equality. We then introduce:
\begin{corollary}[Paul Erd\H{o}s' Corollary {\cite[pp.~305]{infPlanar}}]
    $\\$
    If every finite subgraph of $G$ is planar, $G$ is planar.
\end{corollary}
As edges must connect two vertices and it is self-evident the removal of an edge or vertex on a planar graph cannot affect planarity, it is then sufficient to show that every finite induced subgraph is planar. As such, as $K_{1, \infty}[x_1, \dots, x_n] \subseteq K_{1, n-1}$, $K_{1,\infty}$ is planar.

\subsubsection{Infinite Graphs}
$\\$
The order of a graph $G$ is the cardinality of the vertex set.
That is $\left| V(G)\right|=n$.
A graph is then infinite when $\left|V(G)\right|$ or $\left|E(G)\right|$ is infinite.
A popular method to organize a countably infinite graph $G$ is via an increasing union of finite subgraphs~~\cite{diestel}. \\
That is, for a vertex set:
\[V(G) = \{x_1, x_2, \cdots\}\]
For each $n\in\N$ let the finite induced subgraph of $G$ on $n$ vertices be:
\[G_n \coloneqq G[x_1, \cdots, x_n]\]
It then follows:
\begin{align*}
    G_1 \subseteq &G_2 \subseteq \cdots \\
    G_\infty &= \bigcup _{n\in\N} G_n
\end{align*}
This method of organization, however, is not true for all graph sequences, but can be meaningful to be able to describe the behavior of a graph as it approaches a larger limit graph.
%One trivial example is to merely assume a graph whose edges change in respect to the modulo of the order.
When described as such, every finite induced subgraph of $G_\infty$ is in some $G_n$~\cite[pp.~233]{diestel}.
Allowing $G_\infty$ to be perceived as a graph of arbitrary scale.
Unsurprisingly, it follows that if $G_\infty$ of nested $G_n$ is planar for all n, $G_{\infty}$ is also planar as there is no induced subgraph that can contradict planarity.
As it relates to planarity, the far more interesting object is the topological closure of an infinite graph.\\\\
Denoted via a bar over the top, the closure of an infinite graph is the union of the infinite graph and its limit points~\cite[pp.~97]{munkres}.\[\overline{G_\infty} = G_\infty \cup G_\infty'\]
Limit points, often denoted with an apostrophe, are the set of points approached as sets approach infinity, with some points being a literal representation of $\infty$. \\
For example: \[\overline{\N}=\N\cup\{\infty\}\]
Graphs, however, are very odd objects; being merely a collection of vertices, without inherent spacial coordinates, and edges, being two tuples of vertices.
To which, the closure of an infinite graph is merely that of the graph with an additional vertex: $v_\infty$, or vertices, all with appropriate edges.
Notice that as $\overline{\bigcup_{m=1}^{n} G_m} = \bigcup_{m=1}^{n} \overline{G_m}$ for $n$-finite and $G_m$ being subsets of $G_n$, limit vertices and edges must be added to each appropriate finite and infinite subgraph, and if there are infinitely many subgraphs, as may occur with some trees or many constructions, an additional subgraph $G_{n\to\infty}$ may itself be needed.\\\\
Planarity is a property of the image of a graph.
For the closure of an infinite graph to then be planar, it is self-evident the following equality must hold:
\[\Psi(\overline{G_\infty}) = \overline{\Psi(G_\infty)}\]
That is, $\Psi: G_m \to \R^2$ must be a topological embedding~\cite[Theorem 3.51]{kumar}.
A topological embedding f for topological spaces $(X, T), (Y, T')$ is a homeomorphism from $X$ to $Y$~\cite[Definition 18.4]{munkTop}.
A homeomorphism is a bijection (one-to-one and onto) between topologies: $f: A\to B$, where both $f$ and $f^{-1}$ are continuous~\cite[pp.~105]{munkres}.
$(X, T)$ is a topological space iff $T$ is a topology on $X$~\cite[Definition 12.2]{munkTop}.\\
$T$ is a topology on $X$ iff:~\cite[Definition 12.1]{munkTop}
\begin{enumerate}
    \item $T\subseteq \wp{\left(X\right)}$ ($\wp{\left(X\right)}$ is the power set from discrete mathematics)
    \item $\emptyset \in T$ and $X \in T$
    \item $\forall\ S\subseteq T,\ \cup\ S \in T$
    \item $\forall\ U, V \in T,\ U \cap V \in T$
\end{enumerate}
Notice: $T$ is a set of sets and $\wp{\left(S\right)}$ of any set is a topology.\\\\
\textbf{Or in other words, as $(X, \wp{\left(X\right)})$ is a topological space, a topological embedding is just a function between two domains that is one-to-one, onto, and not discontinuous.}
\newpage

\subsubsection{Planarity of $\overline{K_{1, \infty}}$}
$\\$Let's revisit our chosen construction for $K_{1, n-1}$. \\
The image in $\R^2$ written in radial coordinates: $(r, \theta)$, as sets, follows as:
\begin{align*}
    \Psi\left(V\left(K_{1,n-1}\right)\right) &= \left\{(0,0), \left(r_0, 0\right), \left(r_1, \frac{2\pi}{n-1}\right),  \cdots, \left(r_{n-2}, \frac{2\pi}{n-1}\left(n-2\right)\right)\right\}\\
    \Psi\left(E\left(K_{1,n-1}\right)\right) &= \left\{\left(r_0 t_0, 0\right), \left(r_1 t_1, \frac{2\pi}{n-1}\right),  \cdots, \left(r_{n-2}t_{n-2}, \frac{2\pi}{n-1}\left(n-2\right)\right)\middle|\ 0\leq t_i \leq 1\right\}
\end{align*}
For the sequence of degrees: $\theta_i = \{\frac{2\pi}{n-1} i|\ 0\leq i \leq n-2\}$
\begin{equation}
    \lim_{n\to\infty} \theta_{n-2} = \frac{2\pi}{n-1}(n-2) = 2\pi
\end{equation}
That is, $2\pi$ is a limit point of $\Psi(V(K_{1,n-1}))$ and also therefore a limit point of $K_{1,\infty}$. Thus, under this construction: \[\Psi\left(V\left(\overline{K_{1,\infty}}\right)\right) = \Psi\left(V\left(K_{1,\infty}\right)\right) \cup \{\left(r_{n-1}, 2\pi\right), \dots\}\]
That is, the points $(r_0, 0)$ and $(r_{n-1}, 2\pi)$ are in the closure of this embedding. As both, $r_0, r_{n-1} > 0$ and have edges to the origin. This would imply the edges intersect at all radii from $\min\{r_0, r_{n-1}\} \to 0$ \textbf{for this construction}.\\\\
Consider, however:
\begin{align*}
    \Psi\left(V\left(K_{1,n-1}\right)\right) &= \left\{(0,0), \left(r_0, 0\right), \left(r_1, \frac{2\pi-\epsilon}{n-1}\right),  \cdots, \left(r_{n-2}, \frac{2\pi-\epsilon}{n-1}\left(n-2\right)\right)\right\}\\
    \Psi\left(E\left(K_{1,n-1}\right)\right) &= \left\{\left(r_0 t_0, 0\right), \left(r_1 t_1, \frac{2\pi-\epsilon}{n-1}\right),  \cdots, \left(r_{n-2}t_{n-2}, \frac{2\pi-\epsilon}{n-1}\left(n-2\right)\right)\middle|\ 0\leq t_i \leq 1\right\}
\end{align*}
The sequence of degrees is then: $\theta_i =\{\frac{2\pi-\epsilon}{n-1}i|\ 0\leq i\leq n-2\}$, with limit:
\begin{equation}
    \lim_{n\to\infty}\theta_{n-2}=\frac{2\pi-\epsilon}{n-1}(n-2) = 2\pi-\epsilon
\end{equation}
Let $\epsilon \to 0$, and the domain of $\theta = [0, 2\pi)\in\R$. With $\theta$ strictly less than $2\pi$, this contradiction cannot occur. It may then be concluded $\overline{K_{1,\infty}}$ has a planar construction.\\\\
Notice: The domain of $\theta$ is actually continous in the real number system. While for any finite n, the values of $\theta$ are evenly spaced. For any rational number between 0 and 2$\pi$ non-inclusive, we can choose the degree closest to it and assign it to a sequence $S_n$ as value $S_i\ \forall i \leq n \in \N$. The result is a convergent infinite sequence of difference certainly less than $2\pi$. By the definition of a real number, the chosen degree is in the domain of our degrees, and if it wasn't, it most certainly is in the closure.
\newpage

\subsection{Planarity of Infinite Trees}

\subsubsection{Planarity of $T_{n, m}$}$\\$
Let $T_{n, m}$ be a tree with a central vertex $v_c$ chosen such that the induced subgraph: \[T_{n, m}[\{d(v_c, v_i) = j \cup d(v_c, v_i) = j-1\ \forall v_i\in V(T_{n, m})\}]\] consists only of $K_{1, n-1}$ stars for $j>1$, and that no vertex in $T_{n, m}$ satisfies: \[\{d(v_c, v_i) > m\ \forall v_i \in V(T_{n,m})\}\]
\begin{figure}[htbp]
    \centering %\linewidth or scale=.3
    \includegraphics[height=.25\textheight]{T_65}
    \caption{Planar image of $T_{6, 5}$}
    \label{fig:screenshot}
\end{figure}
Construction of the image: \\
Base Case: m = 1\\
Let $T_{n, 1}$ be $K_{1, n-1}$
That is, for $\{k_0\in \Z|\ 0\leq k \leq n-1\}$:
\begin{align*}
    \Psi(V(T_{n, 1})) &= \{e^{i\frac{2\pi}{n-1}k_0}\} \cup \{(0, 0)\}\\
    \Psi(E(T_{n, 1})) &= \{t_{k_0}e^{i\frac{2\pi}{n-1}k_0}\ 0\leq t_k\leq 1 \in \R \}
\end{align*}
$T_{n, 1}$ is the star $K_{1, n-1}$; all stars are planar.\\
Inductive Hypothesis:\\
Let $\{k_q = \{\forall k \in \Z|\ 0\leq k \leq n-2\};\ \forall q \leq m \in \N \}$. \\
That is, k indexed on q is a set containing $n-1$ values from $0$ to $n-2$. When used in arithmetic, all combinations of all possible values must be used. \\
Let the image of $T_{n, m}$:
\begin{align*}
    \Psi(V(T_{n, m})) &= \bigcup_{p=0} ^m\left\{\sum_{q=0} ^{p} \left(\frac{\sqrt{2(1-\cos(\frac{2\pi}{n-1}))}}{4}\right)^q(-1)^qe^{i\frac{2\pi}{n-1}k_q}\middle | k_q \neq k_{q-1} \right\}\cup\{(0, 0)\}\\
    \Psi(E(T_{n, m})) &= \bigcup_{p=0} ^m\left\{\left(\frac{-\sqrt{2(1-\cos(\frac{2\pi}{n-1}))}}{4}\right)^pt_{k_p}e^{i\frac{2\pi}{n-1}k_p}+ \sum_{q=0} ^{p-1} \left(\frac{-\sqrt{2(1-\cos(\frac{2\pi}{n-1}))}}{4}\right)^qe^{i\frac{2\pi}{n-1}k_q}\middle | k_q \neq k_{q-1}\right\}
\end{align*}
be assumed planar. That is, let
\[\sum_{q=0} ^{p} \left(\frac{\sqrt{2(1-\cos(\frac{2\pi}{n-1}))}}{4}\right)^q(-1)^qe^{i\frac{2\pi}{n-1}k_q}| k_q \neq k_{q-1}\]
be the set of outermost vertices to appended at the p-th index of m; with the edge set consisting of the last term removed from the sum and added with an unique, individual time component $t_{k_p}$ to create an edge from the p-th outermost vertices to the $(p-1)$-th outermost vertices, be assumed planar.\\\\
We want to show that unioning a new vertex and edge set of the same form will not break planarity.\\
Case 1: Scalar Collision
\begin{quote}
    As we no longer have vertices emanating from a single point, to ensure subsequent stars cannot intersect, a convergent infinite sequence less than half the distance from adjacent vertices is needed.\\\\
    Adding a series of $e^{i\theta}$ for some constant-in-context theta to some $e^{i\frac{2\pi}{n-1}k}$ only acts as a horizontal or vertical translation, thus the primary factor to consider is the scalar series applied to each subsequent star.\\ \\
    First, let us find the distance between two adjecent vertices in $K_{1, n-1}$ with scalar $\lambda$. \\
    It follows that:
    \begin{align*}
        d_{\R^2}(\lambda_m e^0, \lambda_m e^{i\frac{2\pi}{n-1}}) &= \sqrt{\left(\lambda_m-\lambda_m\cos\left(\frac{2\pi}{n-1}\right)\right)^2 + \left(0-\lambda_m\sin\left(\frac{2\pi}{n-1}\right)\right)^2} \\
        &= \lambda_m\sqrt{1 - 2\cos\left(\frac{2\pi}{n-1}\right) + \cos\left(\frac{2\pi}{n-1}\right)^2 + \sin\left(\frac{2\pi}{n-1}\right)^2} \\
        &= \lambda_m\sqrt{2-2\cos\left(\frac{2\pi}{n-1}\right)}\\
        &= \lambda_m\sqrt{2\left(1-\cos\left(\frac{2\pi}{n-1}\right)\right)}
    \end{align*}
    May max-spread be assumed on the scalar radius. That is, may it be assumed no geometric constraints limit the sum of the sequence of scalars or:
    \[\sum_{i=m+1}^\infty \lambda _{i} < \lambda_m\frac{\sqrt{2(1-\cos(\frac{2\pi}{n-1}))}}{2}\]
    Consider the sequence of scalars:
    \[\lambda_m = \left\{\left(\frac{\sqrt{2(1-\cos(\frac{2\pi}{n-1}))}}{4}\middle)^m\right|\ 0\leq m \in \Z\right\}\]
    It follows:
    \begin{align*}
        \sum_{i=m+1} ^\infty \left(\frac{\sqrt{2(1-\cos(\frac{2\pi}{n-1}))}}{4}\right)^i &= \sum _{i=0} ^\infty \left(\frac{\sqrt{2(1-\cos(\frac{2\pi}{n-1}))}}{4}\right)^ i - \sum _{i=0}^m \left(\frac{\sqrt{2(1-\cos(\frac{2\pi}{n-1}))}}{4}\right)^i \\
        &= \frac{1}{1-\frac{\sqrt{2(1-\cos(\frac{2\pi}{n-1}))}}{4}} - \sum _{i=0} ^m \left(\frac{\sqrt{2(1-\cos(\frac{2\pi}{n-1}))}}{4}\right)^i
    \end{align*}
    Reconsidering the inequality:
    \begin{align*}
        \left(1-\frac{\sqrt{2(1-\cos(\frac{2\pi}{n-1}))}}{4}\right)^{-1} - \sum _{i=0} ^m \left(\frac{\sqrt{2(1-\cos(\frac{2\pi}{n-1}))}}{4}\right)^i &< \left(\frac{\sqrt{2(1-\cos(\frac{2\pi}{n-1}))}}{4}\right)^m \cdot \frac{\sqrt{2(1-\cos(\frac{2\pi}{n-1}))}}{2}\\
        \left(1-\frac{\sqrt{2(1-\cos(\frac{2\pi}{n-1}))}}{4}\right)^{-1} - \sum _{i=0}^m \left(\frac{\sqrt{2(1-\cos(\frac{2\pi}{n-1}))}}{4}\right)^i &< 2\left(\frac{\sqrt{2(1-\cos(\frac{2\pi}{n-1}))}}{4}\right)^{m+1}
    \end{align*}
    Let $\lambda = \frac{\sqrt{2(1-\cos(\frac{2\pi}{n-1}))}}{4}$
    \begin{align*}
    (1-\lambda)
        ^{-1}-\sum_{i=0} ^m \lambda ^i &< 2 \lambda ^{m+1}\\
        1 - (1-\lambda) \sum_{i=0} ^m \lambda ^i &< 2 \lambda ^{m+1} (1-\lambda) \\
        1 - 2\lambda ^{m+1} + \lambda ^{m+2} &< \sum_{i=0} ^m \lambda ^i - \sum _{i=0} ^m \lambda ^{i+1} \\
        &< \sum_{i=0} ^m \lambda ^i - \sum _{i=1} ^{m+1} \lambda ^i \\
        &< \lambda ^0 - \lambda ^{m+1}\\
        1-2\lambda^{m+1} + \lambda^{m+2} &< 1 - \lambda^{m+1} \\
        \lambda ^{m+2} - \lambda ^{m+1} &< 0 \\
        \lambda ^{m+1} (\lambda - 1) &< 0
    \end{align*}
    As $0\leq\lambda\leq \frac{1}{2}$, this inequality holds for all $n>2$ and all finite m. \\
    The $n=1$ case is non-graphical; as it implies the union of $K_{1, 0}$ stars or 0 additional vertices.\\
    The $n=2$ case has a $\lambda = 0$, but can be reclaimed through an induced subgraph of the $n=3$ case with the central and a chosen half of the vertices.\\
    It may then be concluded $\lambda _m$ satisfies: \[\sum_{i=m+1} ^\infty \lambda _i < \lambda _m \frac{\sqrt{2(1-\cos(\frac{2\pi}{n-1}))}}{2}\]
\end{quote}
\newpage
Case 2: Self Collision
\begin{quote}
    Under the definition of $T_{n, m}$, $T_{n, m}$ is defined as having a collection of $K_{1, n-1}$ stars for every induced subgraph of graphical distance of signed difference 1. \\
    That is:
    \[T_{n, m}[\{d(v_c, v_i) = j \cup d(v_c, v_i) = j-1\ \forall v_i\in V(T_{n, m})\}]\]
    The optimal way to ensure this property follows is to ensure the union of subsequent $K_{1, n-1}$ stars have exactly 1 vertex intersecting with the edge of its parant star, and to remove the conflicting vertex. \\
    That is, as all previous stars have the net effect solely on translation, may we look at the following:
    \begin{equation*}
        te^{i\frac{2\pi}{n-1}k_1} = e^{i \frac{2\pi}{n-1}k_1} - e^{i\frac{2\pi}{n-1}k_2}
    \end{equation*}
    Notice: By the multiplication of a negative, $-te^{i\frac{2\pi}{n-1}k_1} = -e^{i\frac{2\pi}{n-1}k_1} + e^{i\frac{2\pi}{n-1}k_2}$ will also follow. \\
    Similarly, this intersection is independent of radius, so may the simplist case be observed. That is, let $\lambda = 1$ and $t=0$
    \begin{align*}
        te^{i\frac{2\pi}{n-1} k_1} &= e^{i\frac{2\pi}{n-1}k_1} - e^{i\frac{2\pi}{n-1}k_2}\\
        e^{i\frac{2\pi}{n-1}k_2} &= (1-t)e^{i\frac{2\pi}{n-1}k_1} \\
        &= (1-0) e^{i\frac{2\pi}{n-1}k_1}\\
        \frac{2\pi}{n-1}k_2 &= \frac{2\pi}{n-1}k_1 + 2\pi m \\
        k_2 &= k_1 + 2\pi m (n-1)\\
        m, n &\in \Z \to k_2= k_1
    \end{align*}
    By adding an alternating sign $(-1)^i$ and removing the index of $k_2$ equivalent to $k_1$ or more generally $k_{i} = k_{i-1}$, the properties of $T_{n, m}$ may be preserved without effecting planarity
\end{quote}
Thus, by division into cases, $T_{n, m+1}$ with edges and vertices:
\begin{align*}
    \Psi(V(T_{n, m+1})) &=\Psi(V(T_{n,m})) \bigcup \left\{\sum_{q=0} ^{m+1} \left(\frac{\sqrt{2(1-\cos(\frac{2\pi}{n-1}))}}{4}\right)^q(-1)^qe^{i\frac{2\pi}{n-1}k_q}\middle | k_q \neq k_{q-1} \right\}\\
    \Psi(E(T_{n, m+1})) &= \Psi(V(T_{n, m}))\bigcup \\&\left\{\left(\frac{-\sqrt{2(1-\cos(\frac{2\pi}{n-1}))}}{4}\right)^{m+1}t_{k_{m+1}}e^{i\frac{2\pi}{n-1}k_{m+1}}+ \sum_{q=0} ^{m} \left(\frac{-\sqrt{2(1-\cos(\frac{2\pi}{n-1}))}}{4}\right)^qe^{i\frac{2\pi}{n-1}k_q}\middle | k_i \neq k_{i-1}\right\}
\end{align*}
preserves planarity, and by mathematical induction, $T_{n, m}$ is Planar.
\newpage

\subsubsection{$T_{\infty, \infty}$ is planar}
$\\\\$
To use \textbf{Corollary 2.0.1}; the Paul Erdős Corollary, it must be shown that all finite trees are subgraphs of $T_{n, m}$. \\\\
Let T be a finite tree. \\
Let $v_c$ be any vertex of T.\\
Let m be the maximum graphical distance from $v_c$ and n be the maximum of the number of vertices at each distance $m_i$. That is: \\
Let m = max$\{d(v_c, v_i);\ \forall v_i \in T\}$\\
Let n = max$\{|\{\forall v_i \in V(T)\text{ st }d(v_c, v_i)=m_i\cap v_i \neq v_c\}|;\ \forall m_i \leq m \in \N\}$\\
As there are no vertices of distance greater than m with no more than n vertices existing at each subdistance $m_i$, $T\subseteq T_{n, m}$.\\\\
By \textbf{Corollary 2.0.1}; the Paul Erdős Corollary, the following graphs are planar: \[T_{n, \infty}, T_{\infty, m}, T_{\infty, \infty}\]

\subsection{Remarks about infinite planarity}
$\\$
As should have been evident halfway through the section, there are far-simpler and far-easier means to prove infinite stars and trees are planar. Just use \textbf{Theorem 1.1} or \textbf{Theorem 1.2} properly through induction or even with just the general properties of a type of graph and \textbf{Corollary 2.0.1} will prove it's planar. Rather, there are certain apparent-contradictions I wanted to handle directly.

\subsubsection{A graph may have vertices of infinite degree and remain planar.}
$\\$
That is, it is possible for 2 points to be so close together their distance is immeasurable, yet both points are distinct, their edges do not intersect, and the graph is still planar.

\subsubsection{An infinite graph may be closed and remain planar.}
$\\$
One can then take those points that are immesurably close, note that the limit approaches 0, denote it a limit point in the closure, and said limit point will still be a distinct point from the non-limit points and will not violate planarity.

\subsubsection{An infinite graph may be recursively compacted.}
$\\$
That is, one can take those immesurably close points and still find the room to shove the entire graph into itself between those points an indefinite number of times, and the graph will not be any less planar.\\\\
To some degree, planarity almost seems meaningless. You can take an infinite planar graph of infinite size and of infinite density (vertices per unit area), use a transformation like $f\left(u,v\right)=\left(\frac{1}{\sqrt{2\pi}}\int_{0}^{u}e^{-\frac{x^{2}}{2}}dx,\frac{1}{\sqrt{2\pi}}\int_{0}^{v}e^{-\frac{x^{2}}{2}}dx\right)$ to force the graph to occupy at most a 1x1 unit square making it an infinite planar graph of finite size. By \textbf{Theorem 1.1}, a planar graph has a region and that region will have an area. You can then take a copy of any other planar graph, including itself, and scale it to fit into that region, and as long as you connect it to the graph without violating planarity, you can keep going, and even if you do so infinitely, the resulting graph will still be planar.

\newpage